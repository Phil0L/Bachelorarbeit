\chapter*{Zusammenfassung} % Abstract

Diese Arbeit erweitert das System BPMNGen von 
Weidel~\cite{weidl2024bpmngen} und Shi~\cite{shi2025bpmngen} 
mit neuen Prompting-Strategien zur automatisierten Generierung von 
BPMN 2.0 Prozessmodellen mittels Large Language Models. 
Zunächst wird der ursprüngliche, auf der OpenAI-Assistants-API basierende Ansatz vollständig 
neu strukturiert und in ein objektorientiertes Framework überführt, das unterschiedliche 
Anbieter einheitlich integrieren kann. 
Durch optimierte Instructions, reduzierte Tokennutzunge sowie die Unterstützung mehrerer 
Formate wird die Qualität und 
Konsistenz der erzeugten Diagramme verbessert.
Darüber hinaus führt die Arbeit einen Konversationsmodus auf Basis von 
Chain-of-Thought ein, der interaktive Gespräche, Rückfragen und iterative Diagrammbearbeitung 
ermöglicht. 
Ergänzend werden Funktionen für Datei-Uploads via Base64-Data-URLs, Streaming über 
Server-Sent Events sowie neue Techniken wie Diagramm-Sampling und Reflective Prompting
implementiert.
Eine Performanzanalyse untersucht die vorgeschlagenen Anpassungen auf die Qualität 
der Diagramme, die Geschwindigkeit der Generierung und die Kosten. 
Insgesamt entsteht ein flexibles BPMN Modellierungssystem, 
das den Einsatz von LLMs zur Prozessmodellierung deutlich verbessert.