\chapter{Fazit} % Conclusion

\section{Zusammenfassung}

In dieser Arbeit wurde das bestehende BPMNGen-System grundlegend erweitert und modernisiert, 
um die automatische Erstellung und Bearbeitung von BPMN-Diagrammen mit Hilfe von 
Large Language Models deutlich zu verbessern. 
Dazu wurde zunächst die Architektur vollständig neu strukturiert und in ein objektorientiertes 
System überführt, das mehrere KI-Anbieter flexibel unterstützt. 
Durch optimierte Instructions, neue Interaktionsmodi, das Einbeziehen des Chatverlaufs und 
des aktuellen Diagrammzustands sowie Funktionen wie Datei-Uploads, Streaming 
und die Kombination von Text- und Diagrammausgaben konnte die Qualität der 
Diagramme und die Nutzererfahrung deutlich gesteigert werden. 
Ergänzende Methoden wie Reflective Prompting und Diagramm-Sampling ermöglichen 
zudem eine präzisere und vielfältigere Modellgenerierung. 

Die Analyse zeigt, dass sich die eingesetzten Modelle deutlich hinsichtlich 
Geschwindigkeit, Kosten und Ergebnisqualität unterscheiden. 
Für jeden dieser Aspekte existieren Modelle mit Stärken und Schwächen. 
Ein einzelnes Modell, das in allen Kategorien gleichzeitig die besten Ergebnisse liefert, 
konnte nicht gefunden werden. 
Aus diesem Grund ist die Unterstützung einer breiten Auswahl an Modellen sehr wichtig, 
um je nach Anwendungsfall ein passendes Modell auswählen zu können. 

Insgesamt zeigt die Arbeit, dass moderne Prompting-Strategien und flexible Systemarchitekturen 
einen entscheidenden Beitrag dazu leisten können, LLMs sinnvoll und effizient für die 
BPMN-Modellierung einzusetzen.

\section{Ausblick}

Obwohl das BPMNGen-System nun deutliche Fortschritte erzielt hat, bieten sich zahlreiche 
Ansatzpunkte für zukünftige Entwicklungen. In diesem Abschnitt soll gezeigt werden, inwiefern
das Prompting und die Diagrammerstellung weiter verbessert werden könnte.

\paragraph{Auto layouting} 
Viele generierte Diagramme sind korrekt, aber optisch unübersichtlich.
Man könnte ein KI-Layoutmodell integrieren oder einen heuristischen Ansatz wählen.
Es gibt hierzu bereits Ansätze wie das BPMN-Auto-Layout.
\footnote{\url{https://github.com/bpmn-io/bpmn-auto-layout}}
Dieses kann aber zum aktuellen Stand nicht ohne Überarbeitung zum Projekt hinzugefügt werden.

\paragraph{Automatische Prozessoptimierung}
Der BPMNGen Chatbot könnte auch, ohne dass der Nutzer ihn darum bittet, Vorschläge zur Optimierung machen.
Dies könnten sowohl semantische Vorschläge wie unnötige Elemente, fehlende Gates, etc.\ als auch formale
Vorschläge wie fehlende oder duplizierte IDs oder sogar inhaltliche Vorschläge wie veränderte Prozesse sein.
Der Bot könnte diese Vorschläge ganz ohne Aufforderung zum Beispiel als PopUp anzeigen.

\paragraph{Fine Tuning} Der Chatbot könnte mit den Daten der generierten Diagramme weitertrainiert 
werden. 
Hierbei könnte zum Beispiel mit Hilfe der Änderungen, welche der Nutzer selber durchführt, 
dem Chatbot mitgeteilt werden, was er zukünftig besser machen könnte. 
Alternativ könnten auch 
die Änderungsanweisungen genutzt werden, um dem Chatbot Verbesserungen aufzuzeigen.
Somit wäre es möglich, dass der Bot mit der Zeit immer besser wird.
Es könnte hier sowohl ein generelles Fine-Tuning des BPMNGen Chatbots erstellt werden,
sowie ein nutzerspezifisches Fine-Tuning.

