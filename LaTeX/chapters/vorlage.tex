Fußnoten werden mit dem Befehl \verb|footnote| mitten in den fortlaufenden Text eingefügt. \footnote{Wie man schon im vorherigen Absatz sehen konnte.}

In wissenschaftlichen Arbeiten muss man des öfteren andere Arbeiten zitieren. Dazu nutzt man die Stiloptionen und Zitierbefehle des Pakets \verb+biblatex+, z.\,B.\,\verb|numeric| (=Standard-Stil) oder \verb|verbose| resp. \verb|\cite{name}| oder \verb|\autocite{name}|. In eckigen Klammern kann man noch die Seitenzahl angeben, falls notwendig. Der Name ist ein Schlüssel aus der Datei \verb|bibliography.bib|. Falls einmal ein Werk nur indirekt zu einem Teil der Arbeit beigetragen hat, kann man es auch mit \verb|nocite| angeben, dann landet es in der Literaturliste, ohne dass es im Text ausdrücklich zitert wird.
