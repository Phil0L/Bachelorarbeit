\chapter{Einleitung} % Introduction

\section{Motivation}

Business Process Model and Notation (BPMN) ist ein leistungsfähiges Werkzeug zur 
anschaulichen Darstellung von Geschäftsprozessen. 
Es bietet eine standardisierte Methode zur Visualisierung von Abläufen, erleichtert deren 
Analyse und Optimierung. 
Ein zentrales Problem ist jedoch die Umsetzung von einem Prozess in 
ein Diagramm.
Deshalb werden Prozesse häufig nur in Textform beschrieben, was schnell einen Überblick 
bietet, aber nur schwer zu verstehen und nicht visuell darstellbar ist.

Mit modernen Large Language Models (LLMs) eröffnet sich hier eine neue 
Möglichkeit. 
Diese Modelle können Texte verstehen und in geeignete Formate für Diagramme umwandeln, 
wodurch sowohl Experten als auch Fachfremde bei der Prozessmodellierung unterstützt werden.

Wie dies umgesetzt werden kann zeigte bereits Weidl~\cite{weidl2024bpmngen} in seiner 
Arbeit zur Erstellung von BPMN Diagrammen mit Hilfe von ChatGPT Assistants.
Hierbei wurde ein eigenes JSON Schema entworfen, in dem ChatGPT das Diagramm beschreiben
soll, welches dann zu offiziellem BPMN XML geparst wird um es dann mit einem
BPMN Viewer
\footnote{\url{https://github.com/bpmn-io/bpmn-js}}
anzuzeigen.
Dieses Projekt wurde dann von Shi~\cite{shi2025bpmngen} weiter verbessert.
Dabei wurde eine vollständige Client-Server Architektur implementiert, sowie die 
Instructions des Assistants grundlegend verbessert.

Im bisherigen Projekt zeigen sich jedoch auch einige Möglichkeiten zur Verbesserung. 
Die Diagramme werden nur einmalig erzeugt und können nicht wirklich interaktiv angepasst werden. 
Außerdem ist das Prompting sehr statisch aufgebaut, sodass die KI nicht nachfragen oder auf 
vorherige Nachrichten eingehen kann. 
Das eigene JSON-Format ist zwar funktional, aber fehleranfällig und für die KI schwer zu lernen. 
Auch die Unterstützung weiterer KI-Modelle war nicht vorgesehen, und Funktionen wie Datei-Uploads, 
Streaming oder das gleichzeitige Erzeugen mehrerer Diagramme können noch hinzugefügt werden. 
Diese Arbeit setzt genau hier an und soll das System in diesen Punkten verbessern.

\section{Problemstellung und Zielsetzung}

Das Ziel dieser Arbeit ist es, die Art und Weise zu verbessern, 
wie Nutzende mit einem KI-Modell zusammenarbeiten, um BPMN-Diagramme zu erstellen. 
Frühere Ansätze haben meist nur ein einziges Diagramm erzeugt, ohne die Möglichkeit, 
Rückfragen zu stellen oder gemeinsam Schritt für Schritt an einem Prozess zu arbeiten. 
In der Realität entsteht ein gutes Prozessmodell jedoch selten auf den errsten Versuch. 
Es muss in der Regel angepasst, erweitert und überarbeitet werden. 

Damit die KI bessere Ergebnisse liefern kann, braucht sie jedoch klare Anleitungen und 
wirksame Prompting-Strategien. 
Einfache Anweisungen zur Diagrammerstellung reichen meistens nicht aus, denn das Modell 
muss verstehen, wie sich der Nutzer das Diagramm wirklich vorstellt und dabei 
Unklarheiten erkennen und gegebenenfalls nachfragen können. 
Techniken wie Chain of Thought oder Reflective Prompting helfen dabei, indem sie der KI 
dabei helfen, das gewünschte Ergebins zu erzielen. 
Auch das Erzeugen mehrerer Diagramm-Varianten (Sampling) kann sinnvoll sein, weil so 
vom Nutzenden das beste Ergebnis ausgewählt werden kann.

Hinzu kommt, dass es inzwischen viele verschiedene KI-Modelle gibt, die alle unterschiedliche 
Stärken haben. 
Ein System wie das BPMNGen sollte deshalb mehrere Anbieter unterstützen, damit je nach 
Nutzervorstellung und nach aktuellen Stand der LLMs das passende Modell ausgewählt werden kann. 

Außerdem spielt die Benutzerfreundlichkeit eine wichtige Rolle. 
Funktionen wie das Hochladen von Dateien, das schrittweise Streamen von Antworten oder ein 
dialogartiges Verhalten der KI verbessern die Nutzerefahrung.

Diese Arbeit untersucht daher, wie solche Prompting Methoden, wie man sie von modernen
Chatbots kennt, in BPMNGen integriert werden können, um die KI unterstützte Erstellung und 
Bearbeitung von BPMN-Diagrammen besser zu machen.

Hierfür werden die genannten Verbesserungsideen als Ziel dieser Arbeit gesetzt.

\section{Struktur der Arbeit}

Die Arbeit gliedert sich in mehrere Kapitel, die systematisch aufeinander aufbauen.
Kapitel~2 behandelt die theoretischen Grundlagen, insbesondere BPMN Grundlagen
sowie die verwendeten Prompting- und Interaktionstechniken. 

Kapitel~3 bildet den praktischen Teil der Arbeit. 
Dort wird die bestehende BPMNGen Architektur analysiert und anschließend alle genannten Verbesserungsideen
implementiert. 
Dazu gehört die Umstrukturierung zu einem objektorientierten System, 
das mehrere KI-Anbieter unterstützt und generell besser wartbar und erweiterbar ist. 
Außerdem werden Verbesserungen am Prompting gezeigt, darunter optimierte Instructions, 
neue Modi, das Einbinden von Konversationsverlauf und Diagrammzustand, 
Datei-Uploads mit Base64 Data URLs, Streaming über SSE, die Kombination von Text- und Diagrammausgaben sowie 
Techniken wie Reflective Prompting und Diagramm-Sampling. 

Kapitel~4 untersucht die Performanz des erweiterten Systems. 
Dazu werden die erzeugten Diagramme auf ihre Qualität bewertet, 
etwa im Hinblick auf Vollständigkeit, Konsistenz und Struktur. 
Zusätzlich wird die Geschwindigkeit,
sowie die Kosten der Generierung analysiert, die durch verschiedene Prompting-Strategien, 
Modelle und Formate entstehen. 