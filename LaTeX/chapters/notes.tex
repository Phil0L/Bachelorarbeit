\chapter*{Notizen}

\section*{Gefundene Paper}

\subsection*{A Prompt Pattern Catalog to Enhance Prompt Engineering with ChatGPT}
\begin{itemize}[leftmargin=*]
    \item \textbf{Autoren:} Jules White et al. (2023)
    \item \textbf{Inhalt:} Katalogisierung von Prompting-Mustern zur systematischen Wiederverwendung.
    \item \textbf{Link:} \href{https://arxiv.org/abs/2302.11382}{arxiv.org/abs/2302.11382}
\end{itemize}

\subsection*{Chain-of-Thought Prompting Elicits Reasoning in Large Language Models}
\begin{itemize}[leftmargin=*]
    \item \textbf{Autoren:} Jason Wei et al. (2022)
    \item \textbf{Inhalt:} Einführung der „Chain-of-Thought“-Technik zur Verbesserung logischer Schlussfolgerungen.
    \item \textbf{Link:} \href{https://arxiv.org/abs/2201.11903}{arxiv.org/abs/2201.11903}
\end{itemize}

\subsection*{The Prompt Report: A Systematic Survey of Prompting Techniques}
\begin{itemize}[leftmargin=*]
    \item \textbf{Autoren:} Schulhoff et al. (2024)
    \item \textbf{Inhalt:} Systematische Übersicht mit 58 Techniken, Kategorisierung und Beispielen.
    \item \textbf{Link:} \href{https://arxiv.org/abs/2406.06608}{arxiv.org/abs/2406.06608}
\end{itemize}

\subsection*{A Systematic Survey of Prompt Engineering in Large Language Models}
\begin{itemize}[leftmargin=*]
    \item \textbf{Autoren:} Sahoo et al. (2024)
    \item \textbf{Inhalt:} Überblick über Methoden, Anwendungen, Modelle und Herausforderungen.
    \item \textbf{Link:} \href{https://arxiv.org/abs/2402.07927}{arxiv.org/abs/2402.07927}
\end{itemize}

\subsection*{Training Language Models to Follow Instructions with Human Feedback (InstructGPT)}
\begin{itemize}[leftmargin=*]
    \item \textbf{Autoren:} Ouyang et al. (OpenAI, 2022)
    \item \textbf{Inhalt:} Einsatz von RLHF zur Verbesserung der Instruktionsbefolgung durch Sprachmodelle.
    \item \textbf{Link:} \href{https://arxiv.org/abs/2203.02155}{arxiv.org/abs/2203.02155}
\end{itemize}